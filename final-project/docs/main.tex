\documentclass[a4paper]{article}
\usepackage{amsmath}
\usepackage{geometry}
\usepackage{enumitem}
\usepackage{hyperref}
\usepackage{listings}

\title{(draft) Casual Inference and Discovery: Final Project Report}
\author{Pupipat Singkhorn}
\date{May 2025}

\begin{document}

\maketitle

Dataset: Road accidents list from Y51-58 festival period. \url{https://data.go.th/dataset/item_7d61f508-d2e1-4f0c-8408-dfde29f111f5}

\section*{1. Top Factors Affecting Survivability}
Q: What factors matter the most to the survivability of the person in the accident?\

\textbf{Analysis:} Trained a Random Forest on

\begin{quote}
Age, sex, day\_of\_month, month, hour, road type, status (passenger/pedestrian), injured vehicle, counterpart vehicle, safety measure, and alcohol use
\end{quote}

\noindent and extracted feature importances.

\textbf{Top 5 Predictors:}
\begin{enumerate}
    \item \textbf{Age} (0.328)
    \item \textbf{Hour of day} (0.232)
    \item \textbf{Day of month} (0.100)
    \item \textbf{Unknown alcohol use} (0.044)
    \item \textbf{Sex} (0.021)
\end{enumerate}

\textbf{Interpretation:} Older victims have markedly lower survival rates, followed by the time of day (possibly reflecting traffic or light conditions), then day-of-month effects (first vs. second half of January festival), with unknown alcohol status and gender playing smaller but non-negligible roles.

\section*{2. Helmet Effect in Motorcycle Accidents}
Q: How much does helmet help survivability in motorcycle accident?

\begin{itemize}
    \item \textbf{Sample:} Motorcyclists only, excluding ``unknown'' measures
    \item \textbf{Treatment:} helmet = 1 if ``wore helmet''
    \item \textbf{Method:} DoWhy propensity-score matching
    \item \textbf{Estimate (ATE):} +0.0025 (i.e. +0.25 pp absolute increase in survival)
    \item \textbf{Naïve diff-in-means:} +0.0076 (0.76 pp)
    \item \textbf{Robustness checks:} Placebo treatment $p = 0.84$; random common cause $p = 1.0$
\end{itemize}

\textbf{Interpretation:} Wearing a helmet yields a small but positive survival benefit ($\sim$0.25 percentage points) after adjusting for confounders, though the naïve estimate overstates it (0.76 pp) by failing to account for selection bias.

\section*{3. Seatbelt Effect in Car Accidents}
Q: How much does seatbelt help survivability in car accident?

\begin{itemize}
    \item \textbf{Sample:} Cars (sedan/taxi, pickup, van), excluding unknown measures
    \item \textbf{Treatment:} seatbelt = 1 if ``wore seatbelt''
    \item \textbf{Method:} DoWhy propensity-score matching
    \item \textbf{Estimate (ATE):} +0.0196 (1.96 pp absolute increase)
    \item \textbf{Naïve diff-in-means:} +0.0199 (1.99 pp)
    \item \textbf{Robustness checks:} Placebo $p = 0.93$; random common cause $p = 1.0$
\end{itemize}

\textbf{Interpretation:} Seatbelt use increases survival by about 2 percentage points. Adjusted and naïve estimates are nearly identical, suggesting limited confounding in this subset.

\section*{4. Alcohol's Impact on Survivability}
Q: Does alcohol factor into survivability given the dataset?

\begin{itemize}
    \item \textbf{Sample:} All vehicles, excluding ``unknown'' alcohol status
    \item \textbf{Treatment:} alcohol = 1 if ``drank''
    \item \textbf{Method:} DoWhy propensity-score matching
    \item \textbf{Estimate (ATE):} +0.0064 (0.64 pp)
    \item \textbf{Naïve diff-in-means:} +0.0042 (0.42 pp)
    \item \textbf{Robustness checks:} Placebo $p = 0.70$; random common cause $p = 1.0$
\end{itemize}

\textbf{Interpretation:} Surprisingly, after adjusting for confounders, alcohol consumption appears to increase survival by $\sim$0.6 pp—likely reflecting residual confounding or ``drinker'' correlations (e.g. drinking happens on safer roads/times). The small effect and non-significant refutations ($p > 0.05$) counsel caution.

\section*{5. Hospital Effect on Survivability}
Q: Does the hospital affect the survivability?

\begin{itemize}
    \item \textbf{Model:} Logistic regression predicting survival from demographics, road and crash factors
    \item \textbf{Metric:} For each hospital (by ID), compare \textbf{observed} vs. \textbf{expected} survival rate
    \item \textbf{Result:} Standard deviation of hospital-level survival difference = \textbf{0.0538}
\end{itemize}

\textbf{Interpretation:} Hospitals differ in their mortality outcomes by roughly $\pm$5 pp (one standard-deviation spread). This suggests meaningful variability in care quality or patient mix across hospitals.

\section*{6. Hour-of-Day Survival Patterns (Exploratory)}
Q: Any other interesting information you can gain from this dataset?

\begin{itemize}
    \item \textbf{Plot:} Survival rate by hour (0--23) shows a trough around 4 am ($\sim$96.2\%) and a peak around 12 noon--1 pm ($\sim$98.9\%), then a gradual decline into the evening.
\end{itemize}

\textbf{Insight:} Very early-morning accidents (around dawn) have the lowest survival—perhaps reflecting emergency-response delays or impaired visibility—whereas midday incidents see the highest survival.

\section*{Overall Conclusions}

\begin{itemize}
    \item \textbf{Demographics} (age, sex) and \textbf{temporal factors} (hour/day) dominate survival risk.
    \item \textbf{Protective measures} (helmets, seatbelts) show modest but real increases in survival.
    \item \textbf{Alcohol} effects are confounded and require deeper investigation.
    \item \textbf{Hospital performance} varies substantially, indicating opportunities for targeted quality improvements.
    \item \textbf{Time-of-day} patterns point to resource-allocation or public-safety messaging opportunities during vulnerable periods (e.g. pre-dawn hours).
\end{itemize}

This causal analysis blends machine-learning--driven discovery (feature importances) with formal causal inference (DoWhy estimands and refutation tests), offering both \textbf{what} matters and \textbf{how much} it matters to accident survivability.

\section*{Appendix}

\end{document}
